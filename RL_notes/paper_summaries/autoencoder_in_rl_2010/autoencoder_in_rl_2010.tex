\documentclass{article}
\usepackage{amsmath}
\usepackage{amssymb}
\usepackage{amsfonts}
\usepackage[T1]{fontenc}
\usepackage{bm}
\usepackage{array}
\usepackage{graphicx}
\usepackage[utf8]{inputenc}

\title{Paper summary: Deep autoencoder neural networks in reinforcement learning}

\begin{document}
\maketitle


\section{Idea in few sentances}

\section{Explanation of the central concept}

\section{Methodology}


\section{Initial rambly notes}
\subsection{Abstract}
It is important that the feature spaces resenmbe existing similarities and spatial relations
between observations, thus enabling useful policy learning.
Several methods to improve the topology of the feature space are proposed.

\subsection{Introduction}
Nothing new.


\subsection{Method}
They re-encode all observations after every encoder update because
the feature space and its semantics are changed.
This also makes the approximation of the Q-function invalid.
But this way, through re-encoding, this is ok.

\subsection{Variations and optimizations}
\begin{enumerate}
		\item sparse networks.
		\item reusing old q-values.
		\item re-train the autoencoder when there's a lot of new observations
				(say their number is the same as that of the older ones)
\end{enumerate}

\subsection{Other stuff}






\end{document}
