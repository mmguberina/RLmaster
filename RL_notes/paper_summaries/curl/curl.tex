\documentclass{article}
\usepackage{amsmath}
\usepackage{amssymb}
\usepackage{amsfonts}
\usepackage[T1]{fontenc}
\usepackage{bm}
\usepackage{array}
\usepackage{graphicx}
\usepackage[utf8]{inputenc}

\title{Paper summary: Contrastive unsupervised representations for reinforcement learning (CURL)}

\begin{document}
\maketitle


\section{Idea in few sentances}



\section{Explanation of the central concept}




\section{Methodology}


\section{Initial rambly notes}


\subsection{Abstract}
CURL extracts high-level features from raw pixels by using contrastive learning
and performs off-policy control on top of the extracted features.
It works well.


\subsection{Introduction}
To make image-based RL more efficient, people tried:
\begin{enumerate}
		\item do auxiliary tasks on observations
		\item learn world models that predict the future
\end{enumerate}
CuRL is in the first camp of methods and it uses contrastive learning to learn the representations.


\subsection{Method}
The architecture works like follows.
A batch of samples is taken from the replay buffer.
A stack of frames (4 for Atari, 3 for continuous control) is as single sample.
Two different image transformations are done on the batch,
but the same transformation is applied accross a single stack of frames.
One augmented batch goes to the query encoder, while the other goes to the key encoder.
Contrastive unsupervised learning is the applied to those encoders.
The batch of feature vectors from the query encoder is then used as an input to the 
reinforcement learning algorithm. Those gradients do go through the query encoder
(altough this isn't strictly necessary, you can sacrifice performance for 
cross-task representations).
Instance discrimination similar to SimCLR, MoCo and CPC is used
and it is performed accross frame stacks.
Momentum encoding similar to Moco is used and that's applied on 
keys.
A bi-linear inner product is used to compare keys and queries.
If I read this right, the key and query network are the same network.



\subsection{Other stuff}






\end{document}
