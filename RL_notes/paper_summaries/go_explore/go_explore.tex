\documentclass{article}
\usepackage{amsmath}
\usepackage{amssymb}
\usepackage{amsfonts}
\usepackage[T1]{fontenc}
\usepackage{bm}
\usepackage{array}
\usepackage{graphicx}
\usepackage[utf8]{inputenc}

\title{Paper summary: Go-explore: a new approach for hard-exploration problems (but also First return, then explore)}

\begin{document}
\maketitle


\section{Idea in few sentances}



\section{Explanation of the central concept}




\section{Methodology}


\section{Initial rambly notes}

\subsection{Abstract}
The hypothesis is the following:
the main impediments for effective exploration originate from algorithms forgetting
how to reach previously visited states (\textit{detachment}) and failing to first return to a state
before exploring it (\textit{derailment}).
Go-explore deals with this.

It does so by ``remembering'' promising states and returning to those states before intentioally exploring.
\subsection{Introduction}

\subsection{Method}

\subsection{Other stuff}






\end{document}
